\documentclass[]{beamer}
\usetheme{Dresden}
% \useoutertheme{split}

\usepackage{color}
\usepackage{graphicx}
\usepackage{listings}
\usepackage{lmodern} %% allow bold keywords
\usepackage{menukeys}
\usepackage{qtree}

\definecolor{darkgreen}{rgb}{0,0.5,0}
\definecolor{lightblue}{rgb}{0.2,0.2,1}

\lstset{language=Java,
	basicstyle=\ttfamily\footnotesize,
	keywordstyle=\color{purple},
	commentstyle=\color{darkgreen},
	numberstyle=\tiny\color{gray},
	stringstyle=\color{blue},
	tabsize=4,
	showstringspaces=false,
	breaklines=true,
	keepspaces=true,
	numbers=left,
	escapechar=@
}

\title{Java}
\subtitle{Regular Expressions}
\author{FSR Informatik}
\date{\today}

\begin{document}

\begin{frame}
\titlepage
\end{frame}

\begin{frame}{Overview}
\tableofcontents
\end{frame}

\section{String}
\subsection{Java Classes}
\begin{frame}{Class String}
	String is a Java class representing text.
	\vfill
	Strings are immutable. If you use methods to \emph{change} a string you actually get another string.
	\vfill
	Javadoc: \\
	\scriptsize
	\url{https://docs.oracle.com/javase/7/docs/api/java/lang/String.html}
\end{frame}

\begin{frame}[fragile]{String equality - 1}
	\begin{lstlisting}[basicstyle=\ttfamily\scriptsize]
	String x = "hello";
	String y = "hello";
	if (x.equals(y)) {
	    System.out.println("x equals y");
	}
	if (x == y) {
	    System.out.println("x == y");
	}
	\end{lstlisting}
	\vfill
	Obviously x equals y because they are containing the same text. \\
	(x == y) is true as well because x and y are references to the same object.
\end{frame}

\begin{frame}[fragile]{String equality - 2}
	\begin{lstlisting}[basicstyle=\ttfamily\scriptsize]
	String u = new String("hello");
	String v = new String("hello");
	if (u.equals(v)) {
	    System.out.println("u equals v");
	}
	if (u == v) {
	    System.out.println("u == v");
	}
	\end{lstlisting}
	\vfill
	u also equals v. \\
	But u and v are not references to the same String object.
	Therefore (u == v) is false.
\end{frame}

\begin{frame}{Other Classes}
	Java offers multiple classes to deal with strings according to your needs.
	\vfill
	\textbf{StringBuilder} offers extensive append methods to build up a string.\\
	\scriptsize
	\url{https://docs.oracle.com/javase/7/docs/api/java/lang/StringBuilder.html}
	\normalsize
	\vfill
	\textbf{StringBuffer} is a thread-safe implementation. 
	It offers similar methods to StringBuilder. \\
	\scriptsize
	\url{https://docs.oracle.com/javase/7/docs/api/java/lang/StringBuffer.html}
	\normalsize
\end{frame}

\subsection{Console Input}
\begin{frame}[fragile]{BufferedReader}
	\begin{lstlisting}[basicstyle=\ttfamily\scriptsize]
	import java.io.*;

	public class Input {
	
	    public static void main(String[] args) {
	        try (BufferedReader reader = new BufferedReader(new InputStreamReader(System.in))) {
	            System.out.print("input: ");
	            String input = reader.readLine();
	            System.out.println("> " + input);
	        } catch (IOException e) {
	            e.printStackTrace();
	        }
	    }
	}
	\end{lstlisting}
\end{frame}


\begin{frame}[fragile]{BufferedReader - In Detail}
	In a normal try-catch-finally statement you can use the finally block to close your resources.
	Resources must be closed after the program is finished.\\
	The try-with-resource statement will close the resource in normal case and in exceptional case.
	For that the resource must implement the interface \emph{AutoCloseable}.
	\begin{lstlisting}[basicstyle=\ttfamily\scriptsize]
	try (BufferedReader reader = new BufferedReader(new InputStreamReader(System.in))) {
	    
	} catch (IOException e) {
	
	}
	\end{lstlisting}	
	\scriptsize
	\url{http://docs.oracle.com/javase/tutorial/essential/exceptions/tryResourceClose.html}
	\normalsize
\end{frame}

\section{Regular Expressions}
\subsection{Introduction}
\begin{frame}{}
	A \textbf{regular expression} is a search pattern for text. An often used short version is \textbf{regex}.
	\vfill
	You can use regex in many other programming languages and in some text editors.
	\vfill
	See \url{https://en.wikipedia.org/wiki/Regular_expression} \\
	and \url{https://xkcd.com/208/} for more information.
\end{frame}

\begin{frame}[fragile]{Using RegEx in Java}
	You can check if a regex matches a string via:
	\begin{lstlisting}[basicstyle=\ttfamily\scriptsize]
	String example = "Hello World";
		
	String regex = "Hello World";
		
	// this regular expression matches
	if ( example.matches(regex) ) { 
	    System.out.println(regex + " matches " + example);
	}
	\end{lstlisting}
	The regex (search pattern) is also a string.
	\vfill
	An regex matches an identical string.
\end{frame}

\subsection{Regex - Part 1}
\begin{frame}{Case sensitive}
	\begin{tabular}{ c c | c }
		regex & string & matches? \\
		\hline
		hello world & hello world & \textcolor{darkgreen}{yes} \\
		hello & hello world & \textcolor{red}{no} \\
		world & hello world & \textcolor{red}{no} \\
		hello world & hello & \textcolor{red}{no} \\
		Hello World & hello world & \textcolor{red}{no} \\
	\end{tabular}
\end{frame}

\begin{frame}{Dot}
	The dot \textbf{.} matches any charachter.
	\vfill
	\begin{tabular}{ c c | c }
		regex & string & matches? \\
		\hline
		hello & hello & \textcolor{darkgreen}{yes} \\
		hell. & hello & \textcolor{darkgreen}{yes} \\
		hel.o & hello & \textcolor{darkgreen}{yes} \\
		he.o  & hello & \textcolor{red}{no} \\
		he..o & hello & \textcolor{darkgreen}{yes} \\
		..... & hello & \textcolor{darkgreen}{yes} 
	\end{tabular}
\end{frame}

\begin{frame}{Brackets}
	\textbf{[}regex\textbf{]} matches \textbf{one character} according to the regex inside the brackets.
	\vfill
	\begin{tabular}{ c c | c }
		regex & string & matches? \\
		\hline
		hell[o]     & hello & \textcolor{darkgreen}{yes} \\
		hell[aeiou] & hello & \textcolor{darkgreen}{yes} \\
		hell[o][o]  & hello & \textcolor{red}{no} \\
		hell[ap]    & hello & \textcolor{red}{no} \\
		hel[lo][lo] & hello & \textcolor{darkgreen}{yes} \\
		hell[] & hello & ?
	\end{tabular}\\
\end{frame}

\begin{frame}{Brackets}
	Be careful with your regular expressions in general.
	\vfill
	\begin{tabular}{ c c | c }
		regex & string & matches? \\
		\hline
		hell[] & hello & \textcolor{red}{java.util.regex.PatternSyntaxException}
	\end{tabular}
	\vfill
	\emph{Be careful with regex as input.}	
\end{frame}

\begin{frame}{Caret}
	The caret \textbf{\^{}} inside brackets matches the inverse expression.
	\vfill
	\begin{tabular}{ c c | c }
		regex & string & matches? \\
		\hline
		hell[\^{}o] & hello & \textcolor{red}{no} \\
		hell[\^{}e] & hello & \textcolor{darkgreen}{yes} \\
		hell[\^{}eo] & hello & \textcolor{red}{no}
	\end{tabular}
\end{frame}

\begin{frame}{Range}	
	You can set up ranges for matching. Be careful with the order. 
	[A-z] is possible but [a-Z] throws an exception. \\
	The order is: digits followed by captial letters followed by small letters.
	\vfill
	\begin{tabular}{ c c | c }
		regex & string & matches? \\
		\hline
		hell[a-e] & hello & \textcolor{red}{no} \\
		hell[a-z] & hello & \textcolor{darkgreen}{yes} \\
		hell[a-cd-f] & hello & \textcolor{red}{no} \\
		hell[a-mn-z] & hello & \textcolor{darkgreen}{yes} \\
		hell[a-kg-z] & hello & \textcolor{darkgreen}{yes} 
	\end{tabular}
\end{frame}

\begin{frame}{Or}
	You can combine multiple regex with pipe \textbf{\textbar}.
	\vfill
	\begin{tabular}{ c c | c }
		regex & string & matches? \\
		\hline
		hello\textbar bye & hello & \textcolor{darkgreen}{yes} \\
		hello\textbar bye & bye & \textcolor{darkgreen}{yes} \\
		hello\textbar bye\textbar ciao & hello & \textcolor{darkgreen}{yes} \\
		hello\textbar bye & BYE & \textcolor{red}{no} \\
		hello\textbar bye & cat & \textcolor{red}{no}
	\end{tabular}
\end{frame}

\subsection{Regex - Part 2}
\begin{frame}[fragile]{Meta Characters}
	\begin{tabular}{ c | c }
		regex & matches \\
		\hline 
		\textbackslash d & a digit \\
		\textbackslash D & a non digit \\
		\textbackslash w & a normal character \\
		\textbackslash W & a non character \\
		\textbackslash s & a whitespace \\
		\textbackslash S & a non whitespace
	\end{tabular}
	\vfill
	You have to escape a backslash in \emph{String}. 
	Therefore you need two backslashs for your meta character.
	\begin{lstlisting}[basicstyle=\ttfamily\scriptsize]
	String regex = "\\wello" 
	// for the regex: \wello to match "hello";
	\end{lstlisting}
\end{frame}

\begin{frame}{Meta Characters - Examples}
	\begin{tabular}{ c c | c }
		regex & string & matches? \\
		\hline
		h\textbackslash wllo & hello & \textcolor{darkgreen}{yes} \\
		hell\textbackslash D & hello & \textcolor{darkgreen}{yes} \\
		hell\textbackslash W & hello & \textcolor{red}{no} \\
		he\textbackslash slo & hello & \textcolor{red}{no} \\
		he\textbackslash Slo & hello & \textcolor{darkgreen}{yes} \\
		\textbackslash w & \"{a} & \textcolor{red}{no} \\
		a\textbackslash wb & a-b & \textcolor{red}{no} \\
		a\textbackslash wb & a\textunderscore{}b & \textcolor{darkgreen}{yes}
	\end{tabular}
\end{frame}

\begin{frame}{Quantifiers - 1}
	You can use \textbf{+}, \textbf{*} and \textbf{?} as quantifiers
	to match more than one character.
	\vfill
	\begin{tabular}{ c | c }
		regex & matches \\
		\hline 
		regex+ & one or more appearances of regex \\
		a+ & a, aa, aaa, \dots \\
		\hline
		regex* & any appearances of regex  \\
		a* & "", a, aa, \dots  \\
		\hline
		regex? & zero or one appearances of regex  \\
		a? & "", a 
	\end{tabular}
	\vfill
	"" means the empty string.
\end{frame}

\begin{frame}{Quantifiers - 2}
	You can use braces to quantify occurrences.
	\vfill
	\begin{tabular}{ c | c }
		regex & matches \\
		\hline 
		regex$\lbrace n\rbrace$    & n appearances of regex \\
		a$\lbrace 4\rbrace$        & aaaa \\
		\hline
		regex$\lbrace n, m\rbrace$ & at least n and at most m appearances of regex \\
		a$\lbrace 2,4\rbrace$      & aa, aaa, aaaa \\
		\hline
		regex$\lbrace n,\rbrace$   & at least n appearances of regex \\
		a$\lbrace 3,\rbrace$       & aaa, aaaa, aaaaa, \dots
	\end{tabular}
	\vfill
	"" means the empty string.
\end{frame}

\begin{frame}{Quantifiers - Examples}
	\begin{tabular}{ c c c | c }
		regex & equivalent regex & string & matches? \\
		\hline
		hel$\lbrace 1,\rbrace$o & hel+o &
		hello & \textcolor{darkgreen}{yes} \\
		hello$\lbrace 0,1\rbrace$ & hello? & 
		hello & \textcolor{darkgreen}{yes} \\
		hellom$\lbrace 0,1\rbrace$ & hellom? & 
		hello & \textcolor{darkgreen}{yes} \\
		hel$\lbrace 0,\rbrace$o & hel*o & 
		hello & \textcolor{darkgreen}{yes} \\
		hel$\lbrace 2\rbrace$o & \emph{no equivalent} & 
		hello & \textcolor{darkgreen}{yes}
	\end{tabular}
\end{frame}

\begin{frame}{Start and End}
	The caret \textbf{\^{}} matches the begin and the dollar sign \textbf{\${}} matches the end of a string.
	\vfill
	\begin{tabular}{ c c | c }
		regex & string & matches? \\
		\hline
		hello\^{}     & hello & \textcolor{red}{no} \\
		hello\${}     & hello & \textcolor{darkgreen}{yes} \\
		\^{}hello     & hello & \textcolor{darkgreen}{yes} \\
		\^{}hello\${} & hello & \textcolor{darkgreen}{yes} 
	\end{tabular}
\end{frame}

\begin{frame}{Groups}
	Parentheses \textbf{()} can be used to group regex.
	\vfill
	\begin{tabular}{ c c | c }
		regex & string & matches? \\
		\hline
		(hello)*                   & hellohello & \textcolor{darkgreen}{yes} \\
		(hello)$\lbrace 2,\rbrace$ & hellohello & \textcolor{darkgreen}{yes} \\
		(hello)$\lbrace 2,\rbrace$ & hello      & \textcolor{red}{no} \\
		(he[l]*o)+                 & hellohello & \textcolor{darkgreen}{yes} 
	\end{tabular}
\end{frame}

\subsection{More Information}
\begin{frame}{}
	There are more things to know about regular expressions.
	\vfill
	\url{http://docs.oracle.com/javase/tutorial/essential/regex/index.html}
\end{frame}

\end{document}