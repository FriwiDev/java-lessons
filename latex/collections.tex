\documentclass[]{beamer}
\usetheme{Dresden}
% \useoutertheme{split}

\usepackage{color}
\usepackage{graphicx}
\usepackage{listings}
\usepackage{lmodern} %% allow bold keywords
\usepackage{menukeys}
\usepackage{qtree}

\definecolor{darkgreen}{rgb}{0,0.5,0}
\definecolor{lightblue}{rgb}{0.2,0.2,1}

\lstset{language=Java,
	basicstyle=\ttfamily\footnotesize,
	keywordstyle=\color{purple},
	commentstyle=\color{darkgreen},
	numberstyle=\tiny\color{gray},
	stringstyle=\color{blue},
	tabsize=4,
	showstringspaces=false,
	breaklines=true,
	keepspaces=true,
	numbers=left,
	escapechar=@
}

\title{Java 05}
\subtitle{Generics and Collections}
\author{FSR Informatik}
\date{\today}

\begin{document}

\begin{frame}
\titlepage
\end{frame}
\begin{frame}{Overview}
\tableofcontents
\end{frame}

\section{Generics}
\subsection{Example - Generic Class}
\begin{frame}[fragile]{Generic Class Present}
	\textcolor{red}{T} is a generic type. 
	The type parameter \textcolor{red}{T} can represent any class, for example \emph{String}.
	While compiling \textcolor{red}{T} will be substituted with a concrete type argument.
	\begin{lstlisting}[basicstyle=\ttfamily\scriptsize]
	public class Present<@\textcolor{red}{T}@> {
	    
	    private @\textcolor{red}{T}@ present;
	    
	    public Present(@\textcolor{red}{T}@ present) {
	        this.present = present;
	    }
	    
	    public void print() {
	        System.out.println("Present: " + this.present);
	    }
	}
	\end{lstlisting}	
\end{frame}

\begin{frame}[fragile]{Class Book}
	\begin{lstlisting}[escapechar=!]
	public class Book {
	    
	    private String title;
	    
	    public Book(String title) {
	        this.title = title;
	    }
	    
	    @Override
	    public String toString() {
	        return this.title;
	    }
	}
	\end{lstlisting}	
\end{frame}

\begin{frame}[fragile]{Gift-Wrap the Book}
	The generic class \emph{Present} is used with \emph{Book} as type argument
	for the type parameter T in \emph{Present}.
	% gramar omg
	\vfill
	\begin{lstlisting}
	public static void main (String[] args) {
	    
	    Book book = new Book("Harry Potter");	    
	    Present<Book> present = new Present<Book>(book);
	    
	    present.print();
	}
	\end{lstlisting}
\end{frame}

\subsection{Example - Object}
\begin{frame}[fragile]{Class Present}
	This example uses \emph{Object} instead of the generic type T.
	Further there is a new method \texttt{public Object getPresent()}.
	\begin{lstlisting}[basicstyle=\ttfamily\scriptsize]
	public class Present {
	    
	    private Object present;
	    
	    public Present(Object present) {
	        this.present = present;
	    }
	    
	    public void print() {
	        System.out.println("Present: " + this.present);
	    }
	    
	    public Object getPresent() {
	        return this.present;
	    }
	}
	\end{lstlisting}
\end{frame}

\begin{frame}[fragile]{Unwrap the Book}
	After gift-wrap the book there is a present containing that book.
	\vfill
	If we want to access the book in the present via \texttt{getPresent()} 
	we have to \textcolor{red}{cast} it from \emph{Object} to \emph{Book}.
	\begin{lstlisting}
	public static void main (String[] args) {
	    
	    Book book = new Book("Harry Potter");	    
	    Present present = new Present(book);
	    
	    Book newBook = @\textcolor{red}{(Book)}@ present.getPresent();
	}
	\end{lstlisting}
\end{frame}

\begin{frame}[fragile]{Unwrap the Present}
	If the content of the present is unknown then the content can be casted wrong.
	A wrong cast causes an exception at runtime.
	\vfill
	\begin{lstlisting}
	public static void main (String[] args) {
	    
	    Book book = new Book("Harry Potter");	    
	    Present present = new Present(book);
	    
	    Toy toy = (Toy) present.getPresent();
	}
	\end{lstlisting}
	\vfill
	\emph{Do not forget to write a class Toy while testing this example.}
\end{frame}

\begin{frame}[fragile]{Conclusion}
	\textbf{Use generics for more type safety.}
	\vfill
	Compile-time errors are easier to fix than runtime errors.
\end{frame}

\subsection{Restriction}
\begin{frame}[fragile]{Single Bound Restriction}
	You can restrict the variability of generic types with the keyword \textbf{extends}.
	\begin{lstlisting}[basicstyle=\ttfamily\scriptsize]
	public class Present<T extends Toy> {
	    
	    private T present;
	    
	    public Present(T present) {
	        this.present = present;
	    }
	    
	    public void print() {
	        System.out.println("Present: " + this.present);
	    }
	}
	\end{lstlisting}	
\end{frame}

\begin{frame}[fragile]{Example - Class Toy and Class DollHouse}
	\begin{lstlisting}[escapechar=!, basicstyle=\ttfamily\scriptsize]
	public class Toy {
	    
	    private String name;
	    
	    public Toy(String name) {
	        this.name = name;
	    }
	    @Override
	    public String toString() {
	        return this.name;
	    }
	}
	\end{lstlisting}
	\begin{lstlisting}[basicstyle=\ttfamily\scriptsize]
	public class DollHouse extends Toy {
	    
	    public DollHouse(String name) {
	        super(name);
	    }
	}
	\end{lstlisting}
\end{frame}

\begin{frame}[fragile]{Example - Test}
	Only instances of \emph{Toy} or instances of subclasses are allowed in a \emph{Present}.
	Putting a \emph{Book} in a \emph{Present} will cause an error at compile-time.
	\begin{lstlisting}
	public static void main (String[] args) {
	    
	    Present<DollHouse> present1 = 
	        new Present<DollHouse>(new DollHouse("")); 
	    Present<Toy> present2 = 
	        new Present<Toy>(new Toy("car"));
	    
	    present1.print();
	    present2.print();
	}
	\end{lstlisting}
\end{frame}

%% multiple bounds with one class and many interfaces
\begin{frame}[fragile]{Multiple Bound Restriction}
	Let \emph{Playable} and \emph{Giftable} be interfaces and let \emph{Toy} be a class. \\
	T can be substituted with either Toy and its subclasses 
	or with the interfaces \emph{Playable}, \emph{Giftable}, their subinterfaces
	and their implementations as type argument. \\
	Multiple interfaces but \textbf{at most one class} are allowed.
	\begin{lstlisting}[basicstyle=\ttfamily\scriptsize]
	public class Present<T extends Toy & Playable & Giftable> {
	    
	    private T present;
	    
	    public Present(T present) {
	        this.present = present;
	    }
	}
	\end{lstlisting}	
\end{frame}

%TODO methods

%\subsection{Wildcards}
%\begin{frame}[fragile]{Wildcard $<$?$>$}
%	\begin{lstlisting}
%	public static void main (String[] args) {
%
%	    Present<DollHouse> present1 = 
%	        new Present<DollHouse>(new DollHouse("")); 
%	    Present<Toy> present2 = 
%	        new Present<Toy>(new Toy("car"));
%	}
%	\end{lstlisting}
%   Instead of \texttt{Present<Toy>} we can use the wildcard \texttt{Present<?>}.
%	\begin{lstlisting}
%	public static void main (String[] args) {
%
%	    Present<?> present1 = 
%	        new Present<DollHouse>(new DollHouse("")); 
%	    Present<?> present2 = 
%	        new Present<Toy>(new Toy("car"));
%	}
%	\end{lstlisting}
%\end{frame}
%\begin{frame}[fragile]{Wishlist}
%	\begin{lstlisting}
%	public static void main (String[] args) {
%	
%	    Present<?>[] wishlist = new Present<?>[10];
%
%	    wishlist[0] = 
%	        new Present<DollHouse>(new DollHouse("")); 
%	    wishlist[1] = 
%	        new Present<Toy>(new Toy("car"));
%	}
%	\end{lstlisting}
%\end{frame}
%
%\begin{frame}{Guidelines}
%	Oracle offers guidelines for wildcard use:\\
%	\url{https://docs.oracle.com/javase/tutorial/java/generics/wildcardGuidelines.html}
%\end{frame}

\section{Collections}
\subsection{Overview}
\begin{frame}{Collections Framework}
	Java offers various data structures like \textbf{Sets}, \textbf{Lists} and \textbf{Maps}.
	Those structures are part of the collections framework.
	\vfill
	There are interfaces to access the data structures in an easy way.
	There are multiple implementations for various needs.
	Alternatively you can use your own implementations.
	\vfill
	For more information visit:
	\begin{itemize}
		\item \scriptsize\url{https://docs.oracle.com/javase/7/docs/api/java/util/Collection.html}
		\item \url{https://docs.oracle.com/javase/7/docs/api/java/util/Set.html}
		\item \url{https://docs.oracle.com/javase/7/docs/api/java/util/List.html}
		\item \url{https://docs.oracle.com/javase/7/docs/api/java/util/Map.html}
	\end{itemize}
\end{frame}

\subsection{Set and List}
\begin{frame}[fragile]{Set}
	A set is a collection that holds one type of objects.
	A set can not contain one element twice.
	Like all collections the interface \emph{Set} is part of the package \texttt{java.util}.
	\begin{lstlisting}[basicstyle=\ttfamily\scriptsize]
	import java.util.*;

	public class TestSet {
	    
	    public static void main(String[] args) {
	        Set<String> set = new HashSet<String>();
	    
	        set.add("foo");
	        set.add("bar");
	        set.remove("foo");
	        System.out.println(set); // prints: [bar]
	    }
	}
	\end{lstlisting}
	In the following examples \texttt{import java.util.*;} will be omitted.
\end{frame}

\begin{frame}[fragile]{List}
	A list is an ordered collection.
	\vfill
	The implementation \texttt{LinkedList} is a double-linked list.
	\begin{lstlisting}[basicstyle=\ttfamily\scriptsize]
	public static void main(String[] args) {
	
	    List<String> list = new LinkedList<String>();
	    
	    list.add("foo"); 
	    list.add("foo"); // insert "foo" at the end
	    list.add("bar");
	    list.add("foo");
	    list.remove("foo"); // removes the first "foo"
	    
	    System.out.println(list); // prints: [foo, bar, foo]
	}
	\end{lstlisting}
\end{frame}
	
\begin{frame}[fragile]{List Methods}
	some useful List methods:\\
	\vspace{1em}
	\begin{tabular}{ r l r }
		void & \texttt{add(int index, E element)}
		& \footnotesize{insert element at position index} \\
		E &\texttt{get(int index)}
		& \footnotesize{get element at position index} \\
		E &\texttt{set(int index, E element)}
		& \footnotesize{replace element at position index} \\
		E &\texttt{remove(int index)}
		& \footnotesize{remove element at position index}
	\end{tabular}
	\vfill
	some useful LinkedList methods:\\
	\vspace{1em}
	\begin{tabular}{ r l r }
		void & \texttt{addFirst(E element)}
		& \footnotesize{append element to the beginning} \\
		E & \texttt{getFirst()}
		& \footnotesize{get first element} \\
		void & \texttt{addLast(E element)}
		& \footnotesize{append element to the end} \\
		E & \texttt{getLast()}
		& \footnotesize{get last element}
	\end{tabular}
\end{frame}

\begin{frame}[fragile]{Wrapper Class}
	Primitive data types can not be elements in collections. 
	Use wrapper classes like \emph{Integer} instead.
	\begin{lstlisting}[basicstyle=\ttfamily\scriptsize]
	public static void main(String[] args) {
	
	    LinkedList<Integer> list = new LinkedList<Integer>();
	    
	    list.add(3); 
	    list.addFirst(1);
	    list.add(3);
	    list.add(8);
	    list.remove(3); // remove the 4th element
	    list.add(7);
	    
	    System.out.println(list); // prints: [1, 3, 3, 7]
	}
	\end{lstlisting}
\end{frame}

\subsection{Iterating}
\begin{frame}[fragile]{For Loop}
	The for loop can iterate over every element of a collection:\\
	\hspace{1em}\texttt{for (E e : collection)}
	\begin{lstlisting}
	public static void main(String[] args) {
	
	    List<Integer> list = 
	        new LinkedList<Integer>();
	    
	    list.add(1);
	    list.add(3);
	    list.add(3);
	    list.add(7);
	    
	    for (Integer i : list) {
	        System.out.print(i + " "); // prints: 1 3 3 7
	    }
	}
	\end{lstlisting}
\end{frame}

\begin{frame}[fragile]{Iterator}
	An iterator iterates step by step over a collection.
	\begin{lstlisting}[basicstyle=\ttfamily\scriptsize]
	public static void main(String[] args) {
	
	    List<Integer> list = new LinkedList<Integer>();
	    
	    list.add(1);
	    list.add(3);
	    list.add(3);
	    list.add(7);
	    
	    Iterator<Integer> iter = list.iterator();
	    
	    while (iter.hasNext()) {
	        System.out.print(iter.next());
	    }
	    // prints: 1337
	}
	\end{lstlisting}
\end{frame}

\begin{frame}[fragile]{Iterator}
	A standard iterator has only three methods:
	\begin{itemize}
	\item \texttt{boolean hasNext()} - indicates if therer are more elements
	\item \texttt{E next()} - returns the next element
	\item \texttt{void remove()} - returns the current element
	\end{itemize}
	\vspace{1em}
	The iterator is instanced via \texttt{collection.iterator()} :
	\begin{lstlisting}[basicstyle=\ttfamily\scriptsize]
	    Collection<E> collection = new Implementation<E>;
	    Iterator<E> iter = collection.iterator();
	\end{lstlisting}
	Special iterators like \emph{ListIterator} are more sophisticated.
\end{frame}

\subsection{Map}
\begin{frame}[fragile]{Map}
	The interface \emph{Map} is not a subinterface of \emph{Collection}.\\
	A map contains pairs of key and value. Each key refers to a value. 
	Two keys can refer to the same value. There are not two equal keys in one map.
	\emph{Map} is part of the package \texttt{java.util}.
	\vfill
	\begin{lstlisting}[basicstyle=\ttfamily\scriptsize]
	public static void main (String[] args) {
	
	    Map<Integer, String> map = 
	        new HashMap<Integer, String>();
	    
	    map.put(23, "foo");
	    map.put(28, "foo");
	    map.put(31, "bar");
	    map.put(23, "bar"); // "bar" replaces "foo" for key = 23
	    
	    System.out.println(map);
	    // prints: {23=bar, 28=foo, 31=bar}
	}
	\end{lstlisting}
\end{frame}

\begin{frame}[fragile]{KeySet and Values}
	You can get the set of keys from the map.
	Because one value can exist multiple times a collection is used for the values.
	\begin{lstlisting}[basicstyle=\ttfamily\scriptsize]
	public static void main (String[] args) {
	
	    // [...] map like previous slide
	    
	    Set<Integer> keys = map.keySet();
	    Collection<String> values = map.values();
	    
	    System.out.println(keys);
	    // prints: [23, 28, 31]
	    
	    System.out.println(values);
	    // prints: [bar, foo, bar]
	}
	\end{lstlisting}
\end{frame}

\begin{frame}[fragile]{Iterator}
	Therer is no iterator for maps. 
	To iterate over a map use the iterator from the set of keys.
	\begin{lstlisting}[basicstyle=\ttfamily\scriptsize]
	public static void main (String[] args) {
	
	    // [...] map, keys, values like previous slide    	    
	    Iterator<Integer> iter = keys.iterator();
	    
	    while(iter.hasNext()) {
	        System.out.print(map.get(iter.next()) + " ");
	    } // prints: bar foo bar
	    
	    System.out.println(); // print a line break
	    
	    for(Integer i: keys) {
	        System.out.print(map.get(i) + " ");
	    } // prints: bar foo bar
	}
	\end{lstlisting}
\end{frame}

\begin{frame}[fragile]{Nested Maps}
	Nested maps offer storage with key pairs.
	\begin{lstlisting}[basicstyle=\ttfamily\scriptsize]
	public static void main (String[] args) {		
	
	    Map<String, Map<Integer, String>> addresses = 
		    new HashMap<String, Map<Integer, String>>();
		
	    addresses.put("Noethnitzer Str.", 
	        new HashMap<Integer, String>());
		
	    addresses.get("Noethnitzer Str.").
	        put(46, "Andreas-Pfitzmann-Bau");
	    addresses.get("Noethnitzer Str.").
	        put(44, "Fraunhofer IWU");
	}
	\end{lstlisting}
\end{frame}

\end{document}