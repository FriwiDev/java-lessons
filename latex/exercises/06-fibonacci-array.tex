\input{../templates/course_definitions}
\input{../templates/exercise_information}

\title{Exercise}
\subtitle{Fibonacci Array}
\date{\today}


\begin{document}

\begin{frame}
    \titlepage
\end{frame}

\begin{frame}[fragile]{Fibonacci Array}
    A ``FibonacciArray'' can only contain numbers, that are also part of the fibonacci sequence. When other numbers are added, an unchecked ``NoFibonacciException'' is thrown.
\end{frame}

\begin{frame}{Fibonacci Array}
    Tasks:
    \begin{itemize}
        \item Create a class ``FibonacciArray'', that implements the Interface ``Set'' (java.util.Set)
        \item Create a new class ``NoFibonacciException''
        \item Implement a method ``void add(Interger i)'', which performs as described above
        \item Test your implementation with the test class on the next page
    \end{itemize}
\end{frame}

\begin{frame}[fragile]{Fibonacci Array}
   \begin{lstlisting}
public class Test {
    public static void main(String[] args) {
        FibonacciArray fiboMenge = new FibonacciArray();
        for (int i = 0; i < 100; i++) { 
            try {
                fiboMenge.add(i);
                System.out.println("FIB: " + i);
            } catch (NoFibonacciException e) {
                System.out.println("ERR: " + i);
            } 
        }
    } 
}\end{lstlisting}
\end{frame}

\end{document}

