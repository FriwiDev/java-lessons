\input{../templates/course_definitions}
\input{../templates/exercise_information}

\title{Exercise}
\subtitle{Vehicles}
\date{\today}


\begin{document}

\begin{frame}
    \titlepage
\end{frame}

\begin{frame}{Vehicles}
    There are multiple types of vehicles. Cars and Motorcycles are examples. Electric cars are special cars. Each vehicle type has a certain amount of wheels and is manufactured by a manufacturer. Whether taxes have to be paid or not is described by a boolean, which is true by default. Cars can have an autopilot. Electric cars have a battery capacity and do not have to pay taxes.
\end{frame}

\begin{frame}{Vehicles}
    Tasks:
    \begin{itemize}
        \item Model the classes ``Vehicle'', ``Car'', ``ElectricCar'' and ``Motorcycle''
        \item Add the needed attributes and methods
        \item Create constructors, which ask for the vehicle brand and for the electric car, additonally the battery capacity
        \item The method ``toString'' should return the manufacturer and for electric cars, also the battery capacity
        \item Create a class ``Garage'', which creates one instance of each vehicle type (the class contains a main method). Print the name, amount of wheels and the tax status for each vehicle.
    \end{itemize}
\end{frame}

\begin{frame}{Vehicles}
    Tasks:
    \begin{itemize}
        \item The amount of wheels should not be changeable externally
        \item It should be impossible to create an instance of the class ``Vehicle''
    \end{itemize}
\end{frame}

\end{document}

