\documentclass[]{beamer}
\usetheme{Dresden}
% \useoutertheme{split}

\usepackage{color}
\usepackage{graphicx}
\usepackage{listings}
\usepackage{lmodern} %% allow bold keywords
\usepackage{menukeys}
\usepackage{qtree}

\definecolor{darkgreen}{rgb}{0,0.5,0}
\definecolor{lightblue}{rgb}{0.2,0.2,1}

\lstset{language=Java,
	basicstyle=\ttfamily\footnotesize,
	keywordstyle=\color{purple},
	commentstyle=\color{darkgreen},
	numberstyle=\tiny\color{gray},
	stringstyle=\color{blue},
	tabsize=4,
	showstringspaces=false,
	breaklines=true,
	keepspaces=true,
	numbers=left,
	escapechar=@
}

\title{Java}
\subtitle{Abstract}
\author{FSR Informatik}
\date{\today}

\begin{document}

\begin{frame}
\titlepage
\end{frame}

\begin{frame}{Overview}
\tableofcontents
\end{frame}

\section{Abstract}
\subsection{}
\begin{frame}[fragile]{Abstract Class}
	The keyword \textbf{abstract} denotes an abstract class.
	\vfill
	\begin{lstlisting}
	public abstract class AbstractExample {
	
	}	
	\end{lstlisting}
	\vfill
	Like an interface you can not create objects from an abstract class.\\
	Abstract classes can extend other abstract classes and can implement interfaces.\\
	Abstract classes can be extended by normal and abstract classes.
\end{frame}

\begin{frame}[fragile]{Methods}
	An abstract class may has concrete methods and may has abstract methods.
	\begin{lstlisting}
	public abstract class AbstractExample {
	
	    public void printHello() {
	        System.out.println("Hello");	    
	    }
	    
	    public abstract String getName();
	}	
	\end{lstlisting}
	An abstract method forces the class to be abstract as well. \\
	\emph{Methods in an interface are also abstract but not denoted as such.}
\end{frame}

\begin{frame}[fragile]{Subclasses}
	The subclass has to implement abstract methods or has to be abstract as well.
	All concrete methods will be regular inherited.
	\begin{lstlisting}[escapechar=!]
	public class Example extends AbstractExample {
	    
	    @Override
	    public String getName() {
	        return "Example";	    
	    }
	}	
	\end{lstlisting}
\end{frame}

%TODO more text
\begin{frame}{Why using Abstract?}
	Abstract classes are used to minimize similar code in related classes.
\end{frame}

%TODO Overview
%\begin{frame}{Abstract Class vs. Interface}
%
%\end{frame}

%TODO Common Errors
%\begin{frame}{Static vs. Abstract Methods}
%	Do not mix up statitic and abstract methods.
%\end{frame}
\end{document}