\documentclass[]{beamer}
\usetheme{Dresden}
% \useoutertheme{split}

\usepackage{color}
\usepackage{graphicx}
\usepackage{listings}
\usepackage{lmodern} %% allow bold keywords
\usepackage{menukeys}
\usepackage{qtree}

\definecolor{darkgreen}{rgb}{0,0.5,0}
\definecolor{lightblue}{rgb}{0.2,0.2,1}

\lstset{language=Java,
	basicstyle=\ttfamily\footnotesize,
	keywordstyle=\color{purple},
	commentstyle=\color{darkgreen},
	numberstyle=\tiny\color{gray},
	stringstyle=\color{blue},
	tabsize=4,
	showstringspaces=false,
	breaklines=true,
	keepspaces=true,
	numbers=left,
	escapechar=@
}

\title{Java 01}
\subtitle{Introduction}
\author{FSR Informatik}
\date{\today}

\begin{document}

\section{Organisation}
\begin{frame}
	\titlepage
\end{frame}
\begin{frame}{Overview}
	\tableofcontents
\end{frame}

\subsection{Proceeding}
\begin{frame}{About this course}
	Requirements
	\begin{itemize}
		\item You know how to use a computer
		\item Java is not your first programming language or you are a fast learner
	\end{itemize}
	Proceeding
	\begin{itemize}
		\item There will be 10+ lessons
		\item Each covers a topic and comes with excercises
	\end{itemize}
\end{frame}

\subsection{Resources}
\begin{frame}{Some resources}
	\begin{itemize}
		\item You can ask your tutor
		\item Join the Auditorium group \hfill \\
			\url{http://auditorium.inf.tu-dresden.de}
		\item Slides and excercises can be found online \hfill \\
			\url{http://tinyurl.com/slides-java} \\
			\url{http://tinyurl.com/tasks-java}
		\item Official documentation \hfill \\
			\url{http://docs.oracle.com/javase/tutorial/java/nutsandbolts/index.html}
	\end{itemize}
\end{frame}

\section{Your first program}
\subsection{Setting up Eclipse}

\begin{frame}{Receive a copy of Eclipse}
	Eclipse is a powerful IDE\footnote{Integrated Development Environment}, e.g. for Java.
	\begin{itemize}
		\item You can download a Windows version at \\
			\url{http://www.eclipse.org}
		\item Use the package-manager of your Linux distribution
	\end{itemize}
	Eclipse is free and open-source.
\end{frame}

\begin{frame}{Create a new project}
	\begin{enumerate}
		\item \menu[,]{File, New, Java Project}
		\item Name your project \dots
		\item \dots and press the \keys{Finish} button
	\end{enumerate}
\end{frame}

\begin{frame}{Creating a class}
	On the left side is the \emph{Package Explorer}. \\
	It contains a list of all your projects. \\
	The center frame ist the actual editor. \\
	The \emph{Console} and error log for \emph{Problems} is located at the bottom.
	\begin{enumerate}
		\item Right click at your project in the Package Explorer \hfill \\
			\menu[,]{New, Class}
		\item Name your class \\
			\emph{Class names always start with a capital letter!}
		\item Press the \keys{Finish} button
	\end{enumerate}
\end{frame}

\subsection{Hello World!}

\begin{frame}[fragile]{Hello World!}
	This is an empty class after generating:
	\begin{lstlisting}
	public class Hello {
	
	}
	\end{lstlisting}
\end{frame}

\begin{frame}[fragile]{Hello World!}
	This is a small program printing \emph{Hello World!} to the console:
	\begin{lstlisting}
	public class Hello {
	    public static void main(String[] args) {
	        System.out.println("Hello World!");
	    }
	}
	\end{lstlisting}
\end{frame}

\begin{frame}[fragile]{Hello World!}
	\begin{lstlisting}
	public class Hello {
	    public static void main(String[] args) {
	        System.out.println("Hello World!");
	    }
	}
	\end{lstlisting}
	Press \keys{\ctrl + F11} to run the program.
\end{frame}

\begin{frame}[fragile]{Comments}
	\begin{lstlisting}
	public class Hello {
	    // prints a "Hello World!" on your console
	    public static void main(String[] args) {
	        System.out.println("Hello World!");
	    }
	}
	\end{lstlisting}
	You should always comment your code. \\
	Code is read more often than it is written.
	\begin{itemize}
		\item // single line comment
		\item /* comment spanning \\
			multiple lines */
	\end{itemize}
\end{frame}

\section{Basics}
\subsection{Some definitions}

\begin{frame}{Primitive data types}
	Java supports some primitive data types:
	\begin{itemize}
		\item[boolean] a truth value (either \textbf{true} or \textbf{false})
		\item[int] a 32 bit integer
		\item[long] a 64 bit integer
		\item[float] a 32 bit floating point number
		\item[double] a 64 bit floating point number
		\item[char] an ascii character
		\item[void] the empty type (needed in later topics)
	\end{itemize}
\end{frame}

\begin{frame}[fragile]{Blocks}
	\begin{lstlisting}
	public class Hello @\textcolor{red}{\texttt{\{}}@
	    // prints a "Hello World!" on your console
	    public static void main(String[] args) {
	        System.out.println("Hello World!");
	    }
	@\textcolor{red}{\texttt{\}}}@
	\end{lstlisting}
	Everything between \{ and \} is a \emph{block}. \\
	Blocks may be nested.
\end{frame}

\begin{frame}[fragile]{About the Semicolon}
	\begin{lstlisting}
	public class Hello {
	    // prints a "Hello World!" on your console
	    public static void main(String[] args) {
	        System.out.println("Hello World!")@\textcolor{red}{\texttt{;}}@
	    }
	}
	\end{lstlisting}
	Semicolons conclude all statements. \\
	Blocks do not need a semicolon.
\end{frame}

\begin{frame}[fragile]{Naming of Variables}
	\begin{itemize}
		\item The names of variables can begin with any letter or underscore. \\
		Usually the name starts with small letter.
		\item Compound names should use CamelCase.
		\item Use meaningful names.
	\end{itemize}
	\begin{lstlisting}
	public class Calc {
	    public static void main(String[] args) {
	    	int a = 0; // not very meaningful
	    	float myFloat = 5.3f; // also not meaningfull
	    	int count = 7; // quite a good name
	    	
	    	int rotationCount = 7; // there you go
	    }
	}
	\end{lstlisting}
\end{frame}

\subsection{Calculating}

\begin{frame}[fragile, allowframebreaks]{Calculating with \emph{int}}
	\begin{lstlisting}
	public class Calc {
	    public static void main(String[] args) {
	        int a; // declare variable a
	        a = 7; // assign 7 to variable a
	        System.out.println(a); // prints: 7
	        a = 8;
	        System.out.println(a); // prints: 8
	        a = a + 2;
	        System.out.println(a); // prints: 10
	    }
	}
	\end{lstlisting}
	After the first assignment the variable is initialized.
\framebreak
	\begin{lstlisting}
	public class Calc {
	    public static void main(String[] args) {
	        int a = -9; // declaration and assignment of a
	        int b; // declaration of b
	        b = a; // assignment of b
	        System.out.println(a); // prints: -9
	        System.out.println(b); // prints: -9
	        a++; // increments a
	        System.out.println(a); // prints: -8
	    }
	}
	\end{lstlisting}
\framebreak
	\begin{lstlisting}
	public class Calc {
	    public static void main(String[] args) {
	        int b; // declaration of b
	        System.out.println(b);
	    }
	}
	\end{lstlisting}
	Uninitialized variables will cause an Exception. \\
	An Exception is a kind of error we will discuss later.\\
	\vspace{1em}
	\emph{Always assign your variables!}
\framebreak

	Some basic mathematical operations:
	\begin{tabular}{ll}
		Addition & \texttt{a + b;} \\
		Subtraction & \texttt{a - b;} \\
		Multiplication &\texttt{a * b;} \\
		Division & \texttt{a / b;} \\
		Modulo & \texttt{a \% b;} \\
		Increment & \texttt{a++;} \\
		Decrement & \texttt{a--;} \\
	\end{tabular}
\end{frame}

\begin{frame}[fragile, allowframebreaks]{Calculating with \emph{float}}
	\begin{lstlisting}
	public class Calc {
	    public static void main(String[] args) {
	        float a = 9;
	        float b = 7.5f;
	        System.out.println(a); // prints: 9.0
	        System.out.println(b); // prints: 7.5
	        System.out.println(a + b); // prints: 16.5
	    }
	}
	\end{lstlisting}
\framebreak
	\begin{lstlisting}
	public class Calc {
	    public static void main(String[] args) {
	        float a =       8.9f;
	        float b = 3054062.5f;
	        System.out.println(a); // prints: 8.9
	        System.out.println(b); // prints: 3054062.5
	        System.out.println(a + b); // prints: 3054071.5
	    }
	}
	\end{lstlisting}
	Float has a limited precision. \\
	\emph{This might lead to unexpected results!}
\end{frame}

\begin{frame}[fragile]{Mixing \emph{int} and \emph{float}}
	\begin{lstlisting}
	public class Calc {
	    public static void main(String[] args) {
	        float a = 9.3f;
	        int b = 3;
	        System.out.println(a + b); // prints: 12.3
	        float c = a + b;
	        System.out.println(c); // prints: 12.3
	    }
	}
	\end{lstlisting}
	Java converts from \textbf{int} to \textbf{float} by default, if necessary. \\
	But not vice versa.
\end{frame}

\subsection{Text with Strings}

\begin{frame}[fragile]{Strings}
	A String is not a primitive data type but an object. \\
	We discuss objects in detail in the next section.
	\begin{lstlisting}
	public class Calc {
	    public static void main(String[] args) {
	        String hello = "Hello World!";
	        System.out.println(hello); // print: Hello World!
	    }
	}
	\end{lstlisting}
\end{frame}

\begin{frame}[fragile]{Concatenation}
	\begin{lstlisting}
	public class Calc {
	    public static void main(String[] args) {
	        String hello = "Hello";
	        String world = " World!";
	        String sentence = hello + world;
	        System.out.println(sentence);
	        System.out.println(hello + " World!");
	    }
	}
	\end{lstlisting}
	You can concatenate Strings using the +. Both printed lines look the same.
\end{frame}

\begin{frame}[fragile]{Strings and Numbers}
	\begin{lstlisting}
	public class Calc {
	    public static void main(String[] args) {
	    	int factorA = 3;
	    	int factorB = 7;
	    	int product = factorA * factorB;
	    	String answer = 
	            factorA + " * " + factorB + " = " + product;
	        System.out.println(answer); // prints: 3 * 7 = 21
	    }
	}
	\end{lstlisting}
	Upon concatenation, primitive types will be replaced by their current value as \emph{String}.
\end{frame}

\end{document}